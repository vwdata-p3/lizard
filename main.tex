\documentclass{amsproc}

\usepackage{amsrefs}
\usepackage{hyperref}

\newcommand{\doi}[1]{\href{https://doi.org/#1}{doi: #1}} 
\newcommand{\Z}{\mathbb{Z}}
\newcommand{\G}{\mathbb{G}}


\title{Lizard}
\author{Abraham Westerbaan}

\begin{document}
\maketitle
\begin{abstract}
This note specifies
``lizard''
a reversible injection from $\{0,\dotsc,255\}^{16}$
to ristretto255, a prime order elliptic curve group built 
    on top of Curve25519.
\end{abstract}

Curve25519 is a widely adopted elliptic curve
that was originally intended\cite{x25519} for 
Diffie--Hellman key exchange,
but is now also used for
digital signatures\cite{ed25519}
and public key encryption\cite{rfc6637}.
As long as it keeps withstanding attacks and public scrutiny
(as stubbornly as quantum computers resist realisation)
Curve25519 is likely to be the basis for many more popular
cryptographic schemes to come.

From an abstract point of view
Curve25519 provides a cyclic group~$G$
with cheap addition and negation,
but in which less elementary problems, 
such as the computation of discrete logarithms,
remain practically insoluble.
Schemes formulated in terms of~$G$
derive their security primarily from 
the estimated hardness of such problems:
to learn such and such secret,
attackers would
need to solve
this and that problem,
which is currently considered impossible.
While such an abstract view
provides a clean setting 
to formalise schemes
and analyse their security,
an attacker is not bound by it,
and may exploit `implementation details' instead,
of which there are more in an abstract setting.

Care must be taken,
for example,
in checking points provided by untrusted parties,
for if these lie outside the curve,
they may lie in a curve with small order,
possibly enabling an `invalid-curve attack'
(see~\cite{invalidcurveattack,hmqvattack}).
The problem here is not that it is terribly expensive
to check that a point lies on the curve,
but rather that---for the sake of efficiency---this check should be 
performed only once. 
Omission of the check by an
implementor
of a scheme not mentioning such details
is now easily imagined.
Bernstein's Curve25519 Diffie--Hellman scheme cleverly sidesteps
such issues
by an encoding points that 
by encoding points in such a way that 















\begin{bibdiv}
\begin{biblist}
\bib{x25519}{article}{
  title={Curve25519: new {D}iffie--{H}ellman speed records},
  author={Bernstein, Daniel J.},
  conference={
      title={PKC},
      date={2006},
      address={New York}
  },
  book={
      series={LNCS},
      volume={3958}
  },
  pages={207--228},
  year={2006},
  note={\doi{10.1007/11745853\_14}}
}

\bib{ed25519}{article}{
    title={High-speed high-security signatures},
    author={Bernstein, Daniel J.},
    author={Duif, Niels},
    author={Lange, Tanja},
    author={Schwabe, Peter},
    author={Yang, Bo--Yin},
    journal={Journal of Cryptographic Engineering},
    volume={2},
    pages={77--89},
    year={2012},
    note={\doi{10.1007/s13389-012-0027-1}}
}

\bib{decaf}{article}{
    title={Decaf: Eliminating cofactors through point compression},
    book={
        series={LNCS},
        volume={9215}
    },
    author={Hamburg, Mike},
    conference={
        title={CRYPTO},
        date={2015}
    },
    pages={705--723},
    year={2015},
    note={\doi{10.1007/978-3-662-47989-6\_34}}
}

\bib{rfc6637}{report}{
    title={Elliptic Curve Cryptography (ECC) in OpenPGP},
    author={A. Jivsov},
    date={June 2012},
    organization={IETF},
    number={RFC6637},
    note={\doi{10.17487/rfc6637}}
}

\bib{invalidcurveattack}{article}{
    title={Validation of elliptic curve public keys},
    author={Antipa, Adrian},
    author={Brown, Daniel},
    author={Menezes, Alfred},
    author={Struik, Ren{\'e}},
    author={Vanstone, Scott},
    book={
        series={LNCS},
        volume={2567}
    },
    conference={
        title={PKC},
        date={2003}
    },
    pages={211--223},
    year={2003},
    note={\doi{10.1007/3-540-36288-6\_16}}
}

\bib{hmqvattack}{article}{
  title={Another look at HMQV},
  author={Menezes, Alfred},
  journal={JMC},
  volume={1},
  number={1},
  pages={47--64},
  year={2007},
  note={\doi{10.1515/JMC.2007.004}}
}

\end{biblist}
\end{bibdiv}


\section{Notes to self}
ElGamal in a cyclic group is mentioned in \S8.4.2 of 
the ``Handbook of Applied Cryptography''.

Mention Edwards versus Montgomery coordinates.


\end{document}
