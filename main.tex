\documentclass{amsproc}

\usepackage{amsrefs}
\usepackage{hyperref}

\newcommand{\doi}[1]{\href{https://doi.org/#1}{doi: #1}} 
\newcommand{\Z}{\mathbb{Z}}
\newcommand{\G}{\mathbb{G}}


\title{Lizard}
\author{Abraham Westerbaan}

\begin{document}
\maketitle
\begin{abstract}
This note specifies
``lizard''
a reversible injection from $\{0,\dotsc,255\}^{16}$
to ristretto255, a prime order elliptic curve group built 
    on top of Curve25519.
\end{abstract}

Curve25519 is a widely adopted elliptic curve
that was originally intended\cite{x25519} for 
Diffie--Hellman key exchange,
but is now also used for
digital signatures\cite{ed25519}
and public key encryption\cite{rfc6637}.
As long as it keeps withstanding attacks and public scrutiny
(as stubbornly as quantum computers resist realisation)
Curve25519 is likely to be the basis for many more popular
cryptographic schemes to come.

From an abstract point of view
Curve25519 provides a cyclic group~$G$
with cheap addition and negation,
but in which less elementary problems, 
such as the computation of discrete logarithms,
remain practically insoluble.
Schemes formulated in terms of~$G$
derive their security primarily from the hardness of such problems:
to learn such and such secret,
attackers would
need to solve
this and that problem,
which is currently considered impossible.




\begin{bibdiv}
\begin{biblist}
\bib{x25519}{article}{
  title={Curve25519: new {D}iffie--{H}ellman speed records},
  author={Bernstein, Daniel J.},
  conference={
      title={PKC},
      date={2006},
      address={New York}
  },
  book={
      series={LNCS},
      volume={3958}
  },
  pages={207--228},
  year={2006},
  note={\doi{10.1007/11745853\_14}}
}

\bib{ed25519}{article}{
    title={High-speed high-security signatures},
    author={Bernstein, Daniel J.},
    author={Duif, Niels},
    author={Lange, Tanja},
    author={Schwabe, Peter},
    author={Yang, Bo--Yin},
    journal={Journal of Cryptographic Engineering},
    volume={2},
    pages={77--89},
    year={2012},
    note={\doi{10.1007/s13389-012-0027-1}}
}

\bib{decaf}{article}{
    title={Decaf: Eliminating cofactors through point compression},
    book={
        series={LNCS},
        volume={9215}
    },
    author={Hamburg, Mike},
    conference={
        title={CRYPTO},
        date={2015}
    },
    pages={705--723},
    year={2015},
    note={\doi{10.1007/978-3-662-47989-6\_34}}
}

\bib{rfc6637}{report}{
    title={Elliptic Curve Cryptography (ECC) in OpenPGP},
    author={A. Jivsov},
    date={June 2012},
    organization={IETF},
    number={RFC6637},
    note={\doi{10.17487/rfc6637}}
}
\end{biblist}
\end{bibdiv}


\section{Notes to self}
ElGamal in a cyclic group is mentioned in \S8.4.2 of 
the ``Handbook of Applied Cryptography''.

Mention Edwards versus Montgomery coordinates.


\end{document}
