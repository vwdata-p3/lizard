\documentclass{amsproc}

\usepackage{amsrefs}
\usepackage{hyperref}

\newcommand{\doi}[1]{\href{https://doi.org/#1}{doi: #1}} 
\newcommand{\Z}{\mathbb{Z}}
\newcommand{\G}{\mathbb{G}}


\title{Lizard}
\author{Abraham Westerbaan}

\begin{document}
\maketitle
\begin{abstract}
This note specifies
``lizard''
a reversible injection from $\{0,\dotsc,255\}^{16}$
to ristretto255, a prime order elliptic curve group built 
    on top of Curve25519.
\end{abstract}

\section{Background}
Curve25519 is a widely adopted elliptic curve
that was originally intended\cite{x25519} for 
Diffie--Hellman key exchange,
but is now also commonly used for
digital signatures\cite{ed25519}
and public key encryption\cite{rfc6637}.
As long as it keeps withstanding attacks and public scrutiny
(as stubbornly as quantum computers resist realisation)
Curve25519 is likely to be the basis for many more popular
cryptographic schemes to come.

From an abstract point of view
Curve25519 provides a cyclic group~$G$
with cheap addition, negation, and scalar multiplication,
but in which less elementary problems, 
such as the computation of discrete logarithms,
remain practically insoluble.
Schemes formulated in terms of~$G$
derive their security primarily from 
the estimated hardness of such problems:
to learn such and such secret,
attackers would
need to solve
this and that problem,
which is currently considered impossible.
While such an abstract view
provides a clean setting 
to formalise schemes
and analyse their security,
an attacker is not bound by it,
and may exploit `implementation details' instead.

\subsection{Some pitfalls}
Care must be taken,
for example,
in checking that points provided by untrusted parties,
for if these lie outside the curve,
they may lie on a curve of small order,
in which the discrete logarithm problem is tractable
(see~\cite{invalidcurveattack,hmqvattack} for more details.)
The problem here is not that it
is terribly expensive to thwart such 
`invalid-curve attacks'
by checking that points lie on the curve,
but rather that such a check need not be performed on the same point twice,
so that it can easily be omitted once too often by accident
in an implementation of a scheme not concerned with such details.
Bernstein's Curve25519 Diffie--Hellman scheme sidesteps
this issue by using an encoding of points
that never yields points outside the valid curve,
by design.

Another matter to contend with
is the existence of points of order~$8$ in~$G$.
During a Diffie--Hellman key exchange, for instance,
such a point can be used by an attacker
(in a `small-subgroup attack')
to learn the three least significant
bits of its correspondent's secret key.
Although the revelation of these three bits has
negligible impact on the security of vanilla Diffie--Hellman,
these low-order bits are nevertheless prudently set to~$0$
(see~``Responsibilities of the user'', \S3 of~\cite{x25519}).
For more complex protocols,
attacks based on small-order points
can be subtler, and more serious.
The designers of CryptoNote\cite{cryptonote},
for example,
had not taken into account\cite{monero-disclosure,monero-curves-post}
that
their `one-time ring signature',
that was supposed to force the spender of a coin
to reveal the coin's `key image'
(a unique point in~$G$ used to record that the coin was spend),
only forces the spender to reveal
the `key image' \emph{plus} a point of order $\leq 8$,
making it possible to spend the same coin 8~times!

An obvious solution,
used by the CryptoNote based 
currency Monero\cite{monero-fix},
is to restrict
key images
to the first factor of~$G\cong \Z/\ell\Z \oplus \Z/8\Z$
(by checking that the $\ell$-fold sum of the key image is zero,)
so that there is only one key image modulo
the small subgroup $\cong \Z/8\Z$.
Points are not generally restricted to
the large subgroup~$\cong \Z/\ell\Z$ in this way,
because multiplying by a scalar such as~$\ell$
is a rather expensive operation
that can easily ruin the performance of a scheme.
For Monero the impact is relatively minor,
since verifying a `one-time ring signature' with five
public keys already involves over a dozen scalar multiplications,
but only requires a single key-image check.
Restriction to the large subgroup is unsatisfactory in general not only from 
an efficiency standpoint, but also because it reintroduces
the need to check `validity' of points---a check that easily
falls between the cracks.

\subsection{Ristretto}
Mike Hamburg observed that a more
efficient and
unobstrusive
method to remove the small subgroup
would be to change the equality between points,
so that two points are equal when 
their difference is in the small subgroup.













\begin{bibdiv}
\begin{biblist}
\bib{x25519}{article}{
  title={Curve25519: new {D}iffie--{H}ellman speed records},
  author={Bernstein, Daniel J.},
  conference={
      title={PKC},
      date={2006},
      address={New York}
  },
  book={
      series={LNCS},
      volume={3958}
  },
  pages={207--228},
  year={2006},
  note={\doi{10.1007/11745853\_14}}
}

\bib{ed25519}{article}{
    title={High-speed high-security signatures},
    author={Bernstein, Daniel J.},
    author={Duif, Niels},
    author={Lange, Tanja},
    author={Schwabe, Peter},
    author={Yang, Bo--Yin},
    journal={Journal of Cryptographic Engineering},
    volume={2},
    pages={77--89},
    year={2012},
    note={\doi{10.1007/s13389-012-0027-1}}
}

\bib{rfc6637}{report}{
    title={Elliptic Curve Cryptography (ECC) in OpenPGP},
    author={A. Jivsov},
    date={June 2012},
    organization={IETF},
    number={RFC6637},
    note={\doi{10.17487/rfc6637}}
}

\bib{invalidcurveattack}{article}{
    title={Validation of elliptic curve public keys},
    author={Antipa, Adrian},
    author={Brown, Daniel},
    author={Menezes, Alfred},
    author={Struik, Ren{\'e}},
    author={Vanstone, Scott},
    book={
        series={LNCS},
        volume={2567}
    },
    conference={
        title={PKC},
        date={2003}
    },
    pages={211--223},
    year={2003},
    note={\doi{10.1007/3-540-36288-6\_16}}
}

\bib{hmqvattack}{article}{
    title={Another look at HMQV},
    author={Menezes, Alfred},
    journal={JMC},
    volume={1},
    number={1},
    pages={47--64},
    year={2007},
    note={\doi{10.1515/JMC.2007.004}}
}

\bib{cryptonote}{report}{
    title={CryptoNote v~2.0},
    author={van Saberhagen, Nicolas},
    date={2013-10-17},
    note={\url{https://cryptonote.org/whitepaper.pdf}}
}

\bib{monero-disclosure}{misc}{
    title={Disclosure of a Major Bug in CryptoNote Based Currencies},
    author={`luigi1111'},
    author={Riccardo Spagni},
    note={\url{https://web.getmonero.org/2017/05/17/disclosure-of-a-major-bug-in-cryptonote-based-currencies.html}},
    year={2017-05-17}
}

\bib{monero-curves-post}{misc}{
    title={CryptoNote and equivalent points},
    author={Trevor Perrin},
    note={\url{https://moderncrypto.org/mail-archive/curves/2017/000898.html}},
    year={2017-05-18},
}

\bib{monero-fix}{misc}{
    note={\url{https://github.com/monero-project/monero/pull/1744/commits/d282cfcc46d39dc49e97f9ec5cedf7425e74d71f}},
    date={2017-02-20}
}

\bib{decaf}{article}{
    title={Decaf: Eliminating cofactors through point compression},
    book={
        series={LNCS},
        volume={9215}
    },
    author={Hamburg, Mike},
    conference={
        title={CRYPTO},
        date={2015}
    },
    pages={705--723},
    year={2015},
    note={\doi{10.1007/978-3-662-47989-6\_34}}
}

\end{biblist}
\end{bibdiv}


\section{Notes to self}
ElGamal in a cyclic group is mentioned in \S8.4.2 of 
the ``Handbook of Applied Cryptography''.

Mention Edwards versus Montgomery coordinates.

http://safecurves.cr.yp.to

Concretely,
$G$ may be defined by,
writing~$\Z_n$ for the integers modulo~$n$,
\begin{equation*}
G\ =\ \{\ (x,y)\in \Z_q\times \Z_q\colon\  y^2-x^2\,=\,1+dx^2y^2\ \},
\end{equation*}
where~$q:=2^{255}-19$
and $d:=-\frac{121665}{121666}$.
Addition is given by
\begin{equation*}
    (x_1,y_1)\,+\,(x_2,y_2)\ =\  \
    \Bigl(\ \frac{x_1y_2+x_2y_1}{1+dx_1x_2y_1y_2},\ 
    \frac{x_1x_2+y_1y_2}{1-dx_1x_2y_1y_2}\ \Bigr).
\end{equation*}

Writing
\begin{equation*}
    \textstyle
    \alpha \ :=\ \sqrt{1+\frac{\sqrt{1+d}}{d}},
\end{equation*}
the four (odd) order eight points of~$G$
are
\begin{equation*}
(i\alpha,\alpha),\qquad(i\alpha,-\alpha),
    \qquad(-i\alpha,-\alpha),\qquad (-i\alpha,\alpha)
\end{equation*}



\end{document}
